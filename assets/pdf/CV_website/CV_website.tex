%%%%%%%%%%%%%%%%%%%%%%%%%%%% Document Setup %%%%%%%%%%%%%%%%%%%%%%%%%%%%
\documentclass[10pt]{article}
\usepackage[T1]{fontenc}
\usepackage{textcomp}
\usepackage{multicol}
\usepackage{mathptmx}
\usepackage{calc}
\usepackage[shortcuts]{extdash}

%\renewcommand{\familydefault}{\sfdefault}

\reversemarginpar

\usepackage[paper=a4paper,
            marginparwidth=1.2in,   % Length of section titles
            marginparsep=0mm,       % Space between titles and text
            margin=0.9in,           % 25mm margins
           includemp]{geometry}
\setlength{\parindent}{0in}


%\pagestyle{fancy}
%\fancyhf{}\renewcommand{\headrulewidth}{0pt}
%\fancyfootoffset{\marginparsep+\marginparwidth}
%\newlength{\footpageshift}
%\setlength{\footpageshift}
%          {0.5\textwidth+0.5\marginparsep+0.5\marginparwidth-2in}
%\lfoot{\hspace{\footpageshift}%
%       \parbox{4in}{\, \hfill %
%                    \arabic{page} of \protect\pageref*{LastPage}
%                    \hfill \,}}
%                    
                    
%% This gives us fun enumeration environments.
\usepackage{paralist}
\usepackage[shortlabels]{enumitem}
% \usepackage[misc]{ifsym}
%% Reference the last page in the page number
%
% NOTE: comment the +LP line and uncomment the -LP line to have page
%       numbers without the ``of ##'' last page reference)
%
% NOTE: uncomment the \pagestyle{empty} line to get rid of all page
%       numbers (make sure includefoot is commented out above)
%
\usepackage{fancyhdr,lastpage}
\pagestyle{fancy}
%\pagestyle{empty}      % Uncomment this to get rid of page numbers
\fancyhf{}\renewcommand{\headrulewidth}{0pt}
\fancyfootoffset{\marginparsep+\marginparwidth}
\newlength{\footpageshift}
\setlength{\footpageshift}
          {0.1\textwidth+0.1\marginparsep+0.1\marginparwidth-2in}
\lfoot{\hspace{\footpageshift}%
       \parbox{3.5in}{\, \hfill %
                    \arabic{page} of \protect\pageref*{LastPage} % +LP
%                    \arabic{page}                               % -LP
                    \hfill \,}}
                   

% PDF bookmarks
\usepackage{color,hyperref}
\definecolor{darkblue}{rgb}{0.0,0.0,0.3}
\hypersetup{colorlinks,breaklinks,
            linkcolor=darkblue,urlcolor=darkblue,
            anchorcolor=darkblue,citecolor=darkblue}

%%%%%%%%%%%%%%%%%%%%%%%% End Document Setup %%%%%%%%%%%%%%%%%%%%%%%%%%%%


%%%%%%%%%%%%%%%%%%%%%%%%%%% Helper Commands %%%%%%%%%%%%%%%%%%%%%%%%%%%%

%%% HEADING
 \newcommand{\makeheading}[1]%
        {\hspace*{-\marginparsep minus \marginparwidth}%
         \begin{minipage}[t]{\textwidth+\marginparwidth+\marginparsep}%
                {\large #1}\\[.1\baselineskip]%
                 \rule{\columnwidth}{.1pt}%
         \end{minipage}}
         
%\addtolength{\topmargin}{3mm}

%%% SECTION HEADINGS
%% Usage: \section{section name}
%\renewcommand{\section}[1]{\pagebreak[3]%
%    \vspace{1.6\baselineskip}%
%    \phantomsection\addcontentsline{toc}{section}{#1}%
%    \noindent\llap{\scshape\smash{\parbox[t]{\marginparwidth}{\hyphenpenalty=10000\raggedright #1}}}%
%    \vspace{-\baselineskip}\par}

 \renewcommand{\section}[2]%
        {\pagebreak[2]\vspace{1\baselineskip}%
         \phantomsection\addcontentsline{toc}{section}{#1}%
         \hspace{0in}%
         \marginpar{
         \raggedright \scshape #1}#2}

%%% LISTS

% This macro alters a list by removing some of the space that follows the list
% (is used by lists below)
\newcommand*\fixendlist[1]{%
    \expandafter\let\csname preFixEndListend#1\expandafter\endcsname\csname end#1\endcsname
    \expandafter\def\csname end#1\endcsname{\csname preFixEndListend#1\endcsname\vspace{-0.6\baselineskip}}}

% These macros help ensure that items in outer-type lists do not get
% separated from the next line by a page break
% (they are used by lists below)
\let\originalItem\item
\newcommand*\fixouterlist[1]{%
    \expandafter\let\csname preFixOuterList#1\expandafter\endcsname\csname #1\endcsname
    \expandafter\def\csname #1\endcsname{\let\oldItem\item\def\item{\pagebreak[2]\oldItem}\csname preFixOuterList#1\endcsname}
    \expandafter\let\csname preFixOuterListend#1\expandafter\endcsname\csname end#1\endcsname
    \expandafter\def\csname end#1\endcsname{\let\item\oldItem\csname preFixOuterListend#1\endcsname}}
\newcommand*\fixinnerlist[1]{%
    \expandafter\let\csname preFixInnerList#1\expandafter\endcsname\csname #1\endcsname
    \expandafter\def\csname #1\endcsname{\let\oldItem\item\let\item\originalItem\csname preFixInnerList#1\endcsname}
    \expandafter\let\csname preFixInnerListend#1\expandafter\endcsname\csname end#1\endcsname
    \expandafter\def\csname end#1\endcsname{\csname preFixInnerListend#1\endcsname\let\item\oldItem}}

% An itemize-style list with lots of space between items
%
% Usage:
%   \begin{outerlist}
%       \item ...    % (or \item[] for no bullet)
%   \end{outerlist}
% An itemize-style list with lots of space between items
\newenvironment{outerlist}[1][\enskip\textbullet]%
        {\begin{itemize}[#1]}{\end{itemize}%
         \vspace{-0.6\baselineskip}}

% An environment IDENTICAL to outerlist that has better pre-list spacing
% when used as the first thing in a \section
%
% Usage:
%   \begin{lonelist}
%       \item ...    % (or \item[] for no bullet)
%   \end{lonelist}
\newlist{lonelist}{itemize}{3}
    \setlist[lonelist]{label=\enskip\textbullet,leftmargin=*,partopsep=0pt,topsep=0pt}
    \fixendlist{lonelist}
    \fixouterlist{lonelist}

% An itemize-style list with little space between items
%
% Usage:
%   \begin{innerlist}
%       \item ...    % (or \item[] for no bullet)
%   \end{innerlist}
\newlist{innerlist}{itemize}{3}
    \setlist[innerlist]{label=\enskip\textbullet,leftmargin=*,parsep=0pt,itemsep=0pt,topsep=0pt,partopsep=0pt}
    \fixinnerlist{innerlist}

% An environment IDENTICAL to innerlist that has better pre-list spacing
% when used as the first thing in a \section
%
% Usage:
%   \begin{loneinnerlist}
%       \item ...    % (or \item[] for no bullet)
%   \end{loneinnerlist}
\newlist{loneinnerlist}{itemize}{3}
    \setlist[loneinnerlist]{label=\enskip\textbullet,leftmargin=*,parsep=0pt,itemsep=0pt,topsep=0pt,partopsep=0pt}
    \fixendlist{loneinnerlist}
    \fixinnerlist{loneinnerlist}

%%% EXTRA SPACE

% To add some paragraph space between lines.
% This also tells LaTeX to preferably break a page on one of these gaps
% if there is a needed pagebreak nearby.
\newcommand{\blankline}{\quad\pagebreak[3]}
\newcommand{\halfblankline}{\quad\vspace{-0.5\baselineskip}\pagebreak[3]}

%%% FORMATTING MACROS

% Provides a linked \doi{#1} that links doi:#1 to http://dx.doi.org/#1
\usepackage{doi}
% To change the text before the DOI, adjust this command
%\renewcommand\doitext{doi:}

% Provides a linked \url{#1} that doesn't require escape characters
\usepackage{url}

% You can adjust the style \url{} uses here:
% (options are: same, rm, sf, tt; defaults to tt)
\urlstyle{same}

% For \email{ADDRESS}, links ADDRESS to the url mailto:ADDRESS
% (uncomment to typeset the e\-/mail address in typewriter font;
%  otherwise, will be typeset in the \urlstyle above)
%\DeclareUrlCommand\emaillink{\urlstyle{tt}}
\providecommand*\emaillink[1]{\nolinkurl{#1}}
\providecommand*\email[1]{\href{mailto:#1}{\emaillink{#1}}}

\providecommand\BibTeX{{B\kern-.05em{\sc i\kern-.025em b}\kern-.08em \TeX}}
\providecommand\Matlab{\textsc{Matlab}}

% Custom hyphenation rules for words that LaTeX has trouble with
\hyphenation{bio-mim-ic-ry bio-in-spi-ra-tion re-us-a-ble pro-vid-er Media-Wiki}

%%%%%%%%%%%%%%%%%%%%%%%% End Helper Commands %%%%%%%%%%%%%%%%%%%%%%%%%%%

%%%%%%%%%%%%%%%%%%%%%%%%% Begin CV Document %%%%%%%%%%%%%%%%%%%%%%%%%%%%
\begin{document}
\makeheading{Dr. Vidal Lorda, Guillem}


\section{Contact Information}   
\newlength{\rcollength}\setlength{\rcollength}{2.4in}% \rcollength is the width of the right column of the table
\newlength{\spacewidth}\setlength{\spacewidth}{20pt}% \spacewidth is width of area between left and right boxes.
\begin{tabular}[t]{@{}p{\textwidth-\rcollength-\spacewidth}@{}p{\spacewidth}@{}p{\rcollength}}%

% Address box

\parbox[t]{\rcollength}{%

\textit{Email:} \email{gvlorda@gmail.com}\\
%\textit{Website:} \href{http://www.guillemvidal.eu/}{guillemvidal.eu}\\
%\textit{Twitter:} \href{https://twitter.com/guillemvidal_}{@guillemvidal\textunderscore}
}
\end{tabular}

\section{Research Interests}
Political sociology: social class, political attitudes, political conflict.

Political economy: labour markets, welfare states, inequality.

Comparative politics: party competition, voting behaviour, democracy and representation.


\section{Current \\ Position}
2020, Researcher at the Prospective and Strategy Office, Presidency of the Government of Spain.

\vspace{2mm}
%        \begin{innerlist}
%        	\item[] One of the greatest defects of democracy is short-termism. In the frenzy of government, the urgent often overshadows the important. This, in turn, generates other problems - lack of strategic thinking, legislative obsolescence, untapped opportunities, little anticipation, etc. The Prospective and Strategy Office was created to combat this short-termism and ensure the future interests of Spain. It is not about guessing the future, but about systematically analyzing the empirical evidence available to identify the possible challenges and opportunities (demographic, economic, geopolitical, environmental, educational...) that the country will have to face in the medium and long term, and to help prepare for them.
%        \end{innerlist}
        
\section{Professional Experience}
2019 - 2020, Postdoctoral Researcher at the \href{https://wzb.eu/en/research/trans-sectoral-research/center-for-civil-society-research}{Center for Civil Society Research}, \href{https://wzb.eu/en}{WZB Berlin Social Science Center}.
\vspace{2mm}
  \begin{innerlist}
       \item[] The Center for Civil Society Research combines research on political protest and social movements with the empirical analysis of political conflict structuration and social capital. It focuses on three areas: 1) the analysis of long-term changes of political conflict structuring; 2) changes in the organizational landscapes of civil society and the dynamic interaction of different political actors and arenas; and 3) research problems at the interface of social capital and political protest. \\
        \end{innerlist}


2019, Researcher in the \href{https://www.en.gsi.uni-muenchen.de/chairs/comparative/research/dfg-project/index.html}{DfG-EUPOL Project -- European elections and political structuring}, at the \href{https://www.en.gsi.uni-muenchen.de/index.html}{Geschwister Scholl Institute of Political Science, Ludwig Maximilian University (LMU).}

\vspace{2mm}
\begin{innerlist}
        \item[] The project analyzes the relationship between the electoral connection of citizens and parties and the structuring of political conflict in European elections. It aims at examining whether European elections have an independent structuring effect on political conflict and whether this effect has intensified in line with the increasing competencies of the European Parliament. \\
        \end{innerlist}

2014-2019, Researcher in the \href{https://www.eui.eu/Projects/POLCON}{ERC-POLCON Project -- Political Conflict in Europe in the Shadow of the Great Recession}, at the \href{https://www.eui.eu/DepartmentsAndCentres/PoliticalAndSocialSciences}{European University Institute (EUI).}
\vspace{2mm}
        \begin{innerlist}
        \item[] The ERC research program POLCON assesses the contemporary development of European democracies and the politicization of the European integration process in the shadow of the Great Recession. To grasp the political consequences of the economic crisis, the project proposes a combination of a comparative-static analysis of thirty European countries and a dynamic analysis of political conflict in a selected number of cases. It intends to link the study of elections to the study of political protest, covering Western, Southern, as well as Central and Eastern European countries.\\
        \end{innerlist}


2012-2014, Research Assistant to Prof. \href{https://diegomuro.com}{Diego Muro}, at the \href{https://www.ibei.org/en}{Barcelona Institute for International Studies (IBEI-UPF).}

\vspace{2mm}
	 \begin{innerlist}
        \item[--] \href{https://www.ibei.org/en/rules-of-disengagement-individual-and-collective-ways-out-of-terrorism-in-spain-ways-out_19084}{WAYS OUT project} funded by the Spanish Ministry of Education on the ways out of disengagement from terrorism in Spain.
         \item[--] Data Analyst in the Fritz Thyssen Stiftung funded project on the impact of the EU on secessionism in Scotland and Catalonia.
        \end{innerlist}

\clearpage


\section{Education}
PhD in Social and Political Sciences, \href{https://www.eui.eu/DepartmentsAndCentres/PoliticalAndSocialSciences}{European University Institute (EUI).}

\begin{innerlist}
\vspace{1mm}
        \item[] Thesis Topic: \emph{\href{https://cadmus.eui.eu/handle/1814/63265}{The political consequences of the Great Recession in Southern Europe: crisis and representation in Spain}.}
        \item[] Supervisor: Hanspeter Kriesi
        \item[] Defended on June 13, 2019\\
\end{innerlist}

Master of Research in Political and Social Sciences, \href{https://www.eui.eu/DepartmentsAndCentres/PoliticalAndSocialSciences}{European University Institute (EUI).} \\ % 2015

M.A. in International Relations, \href{https://www.ibei.org/en}{Barcelona Institute for International Studies (IBEI-UPF).} % 2013
\vspace{1mm}
\begin{innerlist}
        \item[] Dissertation: \emph{``Its\textquotesingle s the economy, stupid'': A Study of Southern 		European Political Attitudes in the Context of the European Economic Crisis.}
        \item[] Selected Coursework: Comparative Politics, Global Governance, Peace Processes \& 	Conflict Resolution, Poverty, Inequality and Growth.\\
\end{innerlist}

B.Sc. in Economics, \href{https://www.uu.nl/en/organisation/utrecht-university-school-of-economics-use}{Utrecht University School of Economics (USE).} % 2010
\vspace{1mm}
\begin{innerlist}
     \item[] Selected Coursework: Economics of European Integration, Political Economy, Advanced 	Macroeconomics, Econometrics, Economics of the Public Sector.
\end{innerlist}


\section{Visiting \\ Positions}
\href{https://ic3jm.es}{IC3JM - Instituto Carlos III-Juan March de Ciencias Sociales}, Spring 2017. \\

\section{Publications \\ (peer-reviewed)}
Diego Muro, Guillem Vidal, and Martijn Vlaskampf (2019). \href{https://onlinelibrary.wiley.com/doi/full/10.1111/nana.12557}{Does the Prospect of International Recognition Matter for Support for Secession: A Survey Experiment in Catalonia and Scotland}. Nations \& Nationalism, 26(1), 176-96.  \\

Swen Hutter, Hanspeter Kriesi and Guillem Vidal (2018). \href{https://journals.sagepub.com/doi/full/10.1177/1354068817694503}{Old versus New Politics: The Political Spaces in Southern Europe in Times of Crisis.} Party Politics, 24(1), 10-22.\\

Guillem Vidal (2017). \href{https://www.tandfonline.com/doi/abs/10.1080/01402382.2017.1376272}{Challenging business as usual? The rise of new parties in Spain in times of crisis.} West European Politics, 41(2), 261-286. \\

Diego Muro and Guillem Vidal (2016). \href{https://www.tandfonline.com/doi/abs/10.1080/13629395.2016.1168962?journalCode=fmed20}{Political mistrust in southern Europe since the Great Recession.} Mediterranean Politics, 22(2), 197-217.


\section{Book Chapters}
Guillem Vidal and Irene S�nchez-V�tores (2019). \href{https://www.cambridge.org/core/books/european-party-politics-in-times-of-crisis/spain-out-with-the-old-the-restructuring-of-spanish-politics/FA6CDB5B2BA85D849586D91EA5EFABF8#}{"Out with the Old: The Restructuring of Spanish Politics"}. In Hutter, Swen and Hanspeter Kriesi (eds.), European Party Politics in Times of Crisis. Cambridge: Cambridge University Press.\\

Swen Hutter, Argyrios Altiparmakis, and Guillem Vidal (2019). \href{https://www.cambridge.org/core/books/european-party-politics-in-times-of-crisis/diverging-europe-the-political-consequences-of-the-crises-in-a-comparative-perspective/CDDEEC338ADCF06A3680B1BA1EDAEA03}{"Diverging Europe: The Political Consequences of the Crises in a Comparative Perspective."} In Hutter, Swen and Hanspeter Kriesi (eds.), European Party Politics in Times of Crisis. Cambridge: Cambridge University Press.\\

Bj�rn Bremer and Guillem Vidal (2018). \href{https://berghahnbooks.com/title/DoxiadisLiving}{"From boom to bust: A comparative analysis of Greece and Spain under austerity".} In Doxiadis, Evdoxios and Aimee Placas (eds.). Living Under Austerity: Greek Society in Crisis. New York: Berghahn Books.


\section{Working Papers \& Others}
Guillem Vidal and Carlos J. Gil (work-in-progress). Social Class and Vote in Spain. \\

Edgar Grande and Guillem Vidal. \href{https://wzb.eu/system/files/docs/tsr/zz/Discussion%20Paper%20A%20Vote%20for%20Europe.pdf}{A Vote for Europe? The 2019 EP Elections from the Voters' Perspective}. Berlin WZB, Discussion Paper,
ZZ 2020-601. March 2020. \\

Hanspeter Kriesi and Guillem Vidal (work-in-progress). \href{https://www.eui.eu/Documents/DepartmentsCentres/SPS/Profiles/Kriesi/Types-of-Democrats-final.pdf}{Types of Democrats.} \\

Jorge Galindo and Guillem Vidal (work-in-progress). \href{https://www.eui.eu/events/detail?eventid=145024}{Dualism, Welfare State, and the Populist Radical Right.}\\

Berta Barbet, Antoni-�talo De Moragas, and Guillem Vidal (work-in-progress). \href{http://www.ub.edu/polexp/what-the-fact-a-survey-experiment-on-post-truth-politics/}{What the fact? An experiment on the political persuasiveness of experts' advices.} Project financed by Polexp.


\section{Conferences \\ \& Workshops \\ (selection)}
\vspace{-4mm}
\begin{innerlist}
\item[--] CES 26th Annual Conference of Europeanists, Madrid, June 2019.
\item[--] Globalisation, Europe and the Democratic Crisis. London School of Economics, London, UK. February 2018.
\item[--] The Political Consequences of the Great Recession. European University Institute, Florence, Italy. October 2018. 
\item[--] Unchallenged democracy? The political legacy of the economic crisis in Europe. University of York, UK. June 2018. 
\item[--] The Future of Political Parties. Hertie School of Governance, Berlin, Germany. May 2018.
\item[--] CES 23rd Annual Conference of Europeanists, Philadelphia, USA. April 2016.
\item[--] 25th ECPR Summer School on Political Parties and Democracy. University of Leuphana, L�nenburg, Germany. September 2015.
\item[--] UACES 45th Annual Conference, University of Deusto, Bilbao, Spain. September 2015.
\item[--] XII Annual Congress AECPA -Asociaci�n Espa�ola de Ciencia Pol�tica, San Sebasti�n, Spain. July 2015.
\item[--] New Frontiers of Automated Content Analysis in the Social Sciences. University of Zurich (with POLCON project), Zurich, Switzerland. July 2015.
\item[--] Southern European Countries in a Comparative Perspective. IC3JM \& IBEI, Madrid, Spain. November 2014.
\end{innerlist}

\section{Awards \\ \& Grants}
\href{https://www.humboldt-foundation.de/en/}{Postdoctoral Research Fellowships from Alexander von Humboldt Foundation}, 2020.

\vspace{2mm}
Political Sciences Experimental Research Network (Polexp) and Netquest: Survey experimental design award for young researchers, 2016.

\vspace{2mm}
Salvador de Madariaga Grant for Doctoral Studies, 2014-2018, Ministry of Education, Spain.

\section{Outreach}
Op-eds: Social Europe, El Pa�s, European Politics and Policy (EUROPP) LSE Blog, Politikon, Agenda P�blica, Piedras de Papel.

\vspace{2mm}
Mentions in the media: Financial Times, El Pa�s, Infolibre, Trouw.


\section{Languages}
\vspace{-8.3mm}
\begin{multicols}{2}
Spanish, Catalan [native]\\
English, Italian [fluent]

\columnbreak

French [intermediate]\\
Dutch, German [basic]
\end{multicols}
\vspace{-2mm}

\section{Service}
Journal and book reviews: \textit{European Journal of Political Research;
West European Politics;
Government and Opposition;
Party Politics;
South European Society and Politics;
Social Indicators Research;
Manchester University Press.}

\vspace{2mm}
Workshop organizer: \textit{\href{https://www.eui.eu/events/detail?eventid=132835}{Southern Europe in Crisis: A Comparative Perspective}}, May 2017, European University Institute. Two-day workshop with 18 participants. Co-organised with Reto B�rgisser.

\section{Extra Training \\ \& Skills}
Technical workshops:
\begin{innerlist}
\item[--] Causal Inference, March 2018 (Elias Dinas)
\item[--] Panel Data, January 2018 (Hannes Kr�ger)
\item[--] Innovations in Quantitative Content Analysis, February 2015 (Swen Hutter, Wouter van Attefeldt, Peter Makarov)
\item[--] Categorical Data Analysis, January 2015 (Hannes Kr�ger)
\end{innerlist} 
\vspace{2mm}
Teaching: Teacher training certificate, European University Institute, October 2016. 

\vspace{2mm}
Software: Stata, R, LaTex, Office.


\vfill
\section{Last Updated}
January 2021

\end{document}

%%%%%%%%%%%%%%%%%%%%%%%%%% End CV Document %%%%%%%%%%%%%%%%%%%%%%%%%%%%%
